\documentclass{article}

% --------------------
% Packages
% --------------------
\usepackage[utf8]{inputenc}
\usepackage{amsmath, amssymb}
\usepackage{graphicx}
\usepackage{geometry}

% --------------------
% Page Layout
% --------------------
\geometry{
  top=1in,
  bottom=1in,
  left=1in,
  right=1in
}

\setlength{\parindent}{0pt}
\setlength{\parskip}{1em}

% --------------------
% Document
% --------------------
\begin{document}
\vspace*{-1.5cm}

% --------------------
% Header Section
% --------------------
\hspace{0.7cm}
\begin{minipage}[c]{0.7\textwidth}
\vspace{0.5cm}
\textbf{MTech CSE – 1st Semester} \\  
Student ID: \textbf{A125023} \\  
Student Name: \textbf{SRIJITA VERMA}
\end{minipage}
\hspace{0.5cm}
\begin{minipage}{0.2\textwidth}
\centering
\includegraphics[width=0.6\linewidth]{college_logo.jpg}
\end{minipage}

\vspace{1cm}

% --------------------
% Question Section
% --------------------
\section*{Question}
Using the above result, prove that the time complexity of the
\textsc{Build-Heap} algorithm is $O(n)$.

We use the result proved in Question~3(a):

\textit{In a heap containing $n$ elements, the number of nodes at height $h$ is at
most}
\[
N_h \le \left\lceil \frac{n}{2^{h+1}} \right\rceil.
\]


% --------------------
% Answer Section
% --------------------
\section*{Answer}
\section*{Overview of the Approach}

The \textsc{Build-Heap} algorithm constructs a heap from an arbitrary array by
calling the \textsc{Heapify} procedure on internal nodes, starting from the
lowest internal nodes and moving upward to the root.

Although a single call to \textsc{Heapify} can take $O(\log n)$ time in the worst
case, we show that the total cost of all Heapify calls combined is linear, i.e.,
$O(n)$.

This is achieved by:
\begin{itemize}
  \item Grouping nodes by their height,
  \item Bounding the number of nodes at each height using the result from
        Question~3(a),
  \item Summing the total work over all heights.
\end{itemize}

The \textsc{Build-Heap} algorithm constructs the heap in a bottom-up manner by applying \textsc{Heapify} starting from the lowest internal nodes up to the root.
\subsection*{Why \textsc{Build-Heap} Is Not $O(n \log n)$}

A naive analysis might assume that since \textsc{Heapify} takes $O(\log n)$ time
and is called on $O(n)$ nodes, the total running time is $O(n \log n)$. This
argument is incorrect because it ignores the fact that most nodes in a heap are
located at very small heights. The height-based analysis shows that the majority
of \textsc{Heapify} calls take constant or near-constant time, while only a few
nodes incur the maximum cost.

\section*{Preliminaries}

\subsection*{Cost of \textsc{Heapify}}

The \textsc{Heapify} procedure is called on a node and may move the node down the
heap. The maximum number of levels it can move down equals the height of that
node.

Therefore, the running time of \textsc{Heapify} on a node of height $h$ is:
\[
O(h).
\]

\subsection*{Structure of \textsc{Build-Heap}}

The standard \textsc{Build-Heap} algorithm works as follows:
\begin{itemize}
  \item It calls \textsc{Heapify} on all non-leaf nodes.
  \item Leaves have height $0$ and require no work.
  \item Nodes closer to the root have larger height and higher cost per Heapify
        call.
\end{itemize}

Leaf nodes are excluded since they trivially satisfy the heap property and require no heapification.

\section*{Step 1: Group Nodes by Height}

Let:
\begin{itemize}
  \item $n$ be the number of elements in the heap,
  \item $h \ge 0$ denote a height,
  \item $N_h$ be the number of nodes of height $h$.
\end{itemize}

From Question~3(a), we know that:
\[
N_h \le \left\lceil \frac{n}{2^{h+1}} \right\rceil.
\]

\section*{Step 2: Cost Contributed by Nodes at Height $h$}

Each node of height $h$:
\begin{itemize}
  \item Requires $O(h)$ time for \textsc{Heapify}.
\end{itemize}

Thus, the total cost contributed by all nodes at height $h$ is:
\[
O(N_h \cdot h).
\]

Substituting the bound on $N_h$:
\[
O\!\left( \frac{n}{2^{h+1}} \cdot h \right).
\]

\section*{Step 3: Total Cost of \textsc{Build-Heap}}

To obtain the total running time, we sum over all possible heights:
\[
T(n) = \sum_{h=0}^{\lfloor \log n \rfloor}
O\!\left( \frac{n}{2^{h+1}} \cdot h \right).
\]

Factoring out $n$:
\[
T(n) = O\!\left(
n \sum_{h=0}^{\lfloor \log n \rfloor} \frac{h}{2^{h+1}}
\right).
\]

\section*{Step 4: Evaluating the Summation}

Consider the infinite series:
\[
\sum_{h=0}^{\infty} \frac{h}{2^{h+1}}.
\]

This series is convergent and evaluates to a constant. In fact:
\[
\sum_{h=0}^{\infty} \frac{h}{2^{h+1}} = 1.
\]

Since truncating a convergent series only decreases its value, we have:
\[
\sum_{h=0}^{\lfloor \log n \rfloor} \frac{h}{2^{h+1}} \le 1.
\]

So, $\sum_{h=0}^{\infty} \frac{h}{2^{h+1}}$ is a convergent geometric series
weighted by $h$, its value is bounded by a constant independent of $n$.

\section*{Step 5: Final Time Complexity}

Substituting this bound back into the expression for $T(n)$:
\[
T(n) = O(n \cdot 1) = O(n).
\]

\section*{Final Result}



\[
\boxed{\textsc{Build-Heap runs in } O(n) \text{ time.}}
\]

Moreover, this bound is tight, as every element in the input array must be
examined at least once, implying a lower bound of $\Omega(n)$ for heap
construction.

\section*{Remark}

Although the \textsc{Heapify} operation has a worst-case time complexity of
$O(\log n)$, the \textsc{Build-Heap} algorithm runs in linear time because only a
small number of nodes are located at large heights. Most nodes are near the
leaves and require constant or very small time to heapify. The bound on the
number of nodes at each height ensures that the total work performed across all
heap levels is linear in $n$.
\section*{Intuition Behind the Result}

\begin{itemize}
  \item Most nodes in a heap are near the leaves.
  \item Nodes near the leaves have very small height.
  \item Expensive Heapify operations are performed only on a few nodes near the
        root.
  \item The large number of cheap operations dominates the total cost.
  \item Nodes at level $k$ from the root have height approximately $\log n - k$,
      and their number grows exponentially as $k$ increases.
\end{itemize}

This imbalance ensures that the total running time is linear, not $O(n \log n)$.

\section*{Why This Result Is Important}

This analysis explains a key and somewhat counterintuitive fact:

Even though \textsc{Heapify} takes $O(\log n)$ time in the worst case,
\textsc{Build-Heap} runs in $O(n)$ time overall.

This result is fundamental in:
\begin{itemize}
  \item Heap construction,
  \item Heapsort analysis,
  \item Priority queue implementations.
\end{itemize}

\end{document}
