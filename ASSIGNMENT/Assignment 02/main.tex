\documentclass{article}

% --------------------
% Packages
% --------------------
\usepackage[utf8]{inputenc}
\usepackage{amsmath, amssymb}
\usepackage{graphicx}
\usepackage{geometry}

% --------------------
% Page Layout
% --------------------
\geometry{
  top=1in,
  bottom=1in,
  left=1in,
  right=1in
}

\setlength{\parindent}{0pt}
\setlength{\parskip}{1em}

% --------------------
% Document
% --------------------
\begin{document}
\vspace*{-1.5cm}

% --------------------
% Header Section
% --------------------
\hspace{0.7cm}
\begin{minipage}[c]{0.7\textwidth}
\vspace{0.5cm}
\textbf{MTech CSE – 1st Semester} \\  
Student ID: \textbf{A125023} \\  
Student Name: \textbf{SRIJITA VERMA}
\end{minipage}
\hspace{0.5cm}
\begin{minipage}{0.2\textwidth}
\centering
\includegraphics[width=0.6\linewidth]{college_logo.jpg}
\end{minipage}

\vspace{1cm}

% --------------------
% Question Section
% --------------------
\section*{Question}

In an array of size $n$ representing a binary heap, prove that all leaf nodes
are located at indices from $\left\lfloor \dfrac{n}{2} \right\rfloor + 1$ to $n$.

% --------------------
% Answer Section
% --------------------
\section*{Answer}
\section*{Background: Array Representation of a Binary Heap}

A binary heap is a complete binary tree stored in an array using level-order
indexing. For an array indexed from 1 (standard heap convention):

\begin{itemize}
  \item The parent of the node at index $i$ is at index $\left\lfloor \dfrac{i}{2} \right\rfloor$.
  \item The left child of the node at index $i$ is at index $2i$.
  \item The right child of the node at index $i$ is at index $2i + 1$.
\end{itemize}

A leaf node is defined as a node that has no children.

\section*{Key Observation}

A node at index $i$ has at least one child if and only if:
\[
2i \leq n
\]

This is because the left child index is $2i$. If $2i > n$, then both left and
right child indices exceed the array size.

Thus:
\begin{itemize}
  \item Nodes with $2i \leq n$ are internal nodes.
  \item Nodes with $2i > n$ are leaf nodes.
\end{itemize}

\section*{Step-by-Step Proof}

\subsection*{Step 1: Identify the Largest Index of an Internal Node}

We want the largest index $i$ such that:
\[
2i \leq n
\]

Solving:
\[
i \leq \frac{n}{2}
\]

Since array indices are integers:
\[
i \leq \left\lfloor \frac{n}{2} \right\rfloor
\]

Thus, all indices from $1$ to $\left\lfloor \dfrac{n}{2} \right\rfloor$ correspond
to internal nodes.

\subsection*{Step 2: Identify Leaf Node Indices}

Any index $i$ such that:
\[
i > \left\lfloor \frac{n}{2} \right\rfloor
\]

will satisfy:
\[
2i > n
\]

Hence, these nodes cannot have children and are therefore leaf nodes.

\subsection*{Step 3: Determine the Range of Leaf Nodes}

From the above observations:
\begin{itemize}
  \item Internal nodes are at indices $1$ to $\left\lfloor \dfrac{n}{2} \right\rfloor$.
  \item Leaf nodes are at indices $\left\lfloor \dfrac{n}{2} \right\rfloor + 1$ to $n$.
\end{itemize}

\section*{ASCII Tree Diagram for a Binary Heap}

Consider a binary heap with $n = 10$ elements.

\hspace{1.7cm}
\subsection*{Tree Representation}

\begin{verbatim}



                1
              /   \
             2     3
            / \   / \
           4   5 6   7
          / \
         8   9
        /
       10
\end{verbatim}

\subsection*{Corresponding Array Indices}

\begin{center}
\begin{tabular}{l|cccccccccc}
\textbf{Index} & 1 & 2 & 3 & 4 & 5 & 6 & 7 & 8 & 9 & 10 \\
\hline
\textbf{Node}  & R & L1 & L1 & L2 & L2 & L2 & L2 & L3 & L3 & L3
\end{tabular}
\end{center}

\subsection*{Explanation Using the Diagram}

Nodes at indices $1$ to $5$ have at least one child.

For example, node $5$ has a left child at index:
\[
2 \times 5 = 10
\]

Nodes at indices $6$ to $10$ have no children. For any index $i \geq 6$:
\[
2i > 10
\]
and therefore no child exists.

Since:
\[
\left\lfloor \frac{10}{2} \right\rfloor = 5,
\]

we conclude:
\begin{itemize}
  \item Indices $1$ to $5$ correspond to internal nodes.
  \item Indices $6$ to $10$ correspond to leaf nodes.
\end{itemize}

This visually confirms that all leaf nodes occupy indices:
\[
\left\lfloor \frac{n}{2} \right\rfloor + 1 \text{ to } n.
\]

\subsection*{Diagram Explanation (In Words)}


The array representation of a binary heap can be visualized as a complete binary
tree arranged level by level from left to right. The root of the heap corresponds
to index $1$ in the array. The next level of the tree corresponds to indices $2$
and $3$, followed by indices $4, 5, 6,$ and $7$ at the next level, and so on.
Each level of the tree is filled completely before nodes are added to the next
level.

In this representation, each node at index $i$ has its left child at index $2i$
and its right child at index $2i + 1$, provided these indices do not exceed $n$,
the size of the array. Visually, this means that nodes in the upper levels of the
tree always have children, while nodes near the bottom may not.

The last level of the tree is the only level that may be partially filled. All
nodes at this level appear consecutively at the end of the array. These nodes do
not have children because there are no remaining positions in the array to store
them. As a result, they are leaf nodes.

If we observe the array indices, the transition from internal nodes to leaf nodes
occurs exactly after index $\left\lfloor \dfrac{n}{2} \right\rfloor$. Nodes at
indices $1$ through $\left\lfloor \dfrac{n}{2} \right\rfloor$ have at least one
child and therefore correspond to internal nodes in the tree. Nodes at indices
$\left\lfloor \dfrac{n}{2} \right\rfloor + 1$ through $n$ appear in the lowest
level of the tree and have no children, which visually confirms that they are leaf
nodes.

Thus, when the array is interpreted as a level-order traversal of a complete
binary tree, it becomes clear that all leaf nodes are grouped at the end of the
array, occupying indices
$\left\lfloor \dfrac{n}{2} \right\rfloor + 1$ through $n$.

\section*{Illustrative Example}

Let $n = 10$.

\[
\left\lfloor \frac{10}{2} \right\rfloor = 5
\]

\begin{itemize}
  \item Indices $1$ to $5$ are internal nodes.
  \item Indices $6$ to $10$ are leaf nodes.
\end{itemize}

All nodes from index $6$ onward have no children, confirming the result.

\section*{Why This Holds for All Binary Heaps}

This property holds independently of whether the heap is a min-heap or a max-heap,
because it depends solely on:
\begin{itemize}
  \item The complete binary tree structure.
  \item The array indexing scheme.
\end{itemize}

It is not affected by key values or heap order.

\section*{Final Conclusion}

In an array of size $n$ representing a binary heap, all leaf nodes are located at
indices:
\[
\left\lfloor \frac{n}{2} \right\rfloor + 1 \text{ to } n.
\]

This result follows directly from the array representation of a complete binary
tree and the definition of a leaf node.

\end{document}
