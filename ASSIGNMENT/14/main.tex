\documentclass{article}

% --------------------
% Packages
% --------------------
\usepackage[utf8]{inputenc}
\usepackage{amsmath, amssymb}
\usepackage{graphicx}
\usepackage{geometry}

% --------------------
% Page Layout
% --------------------
\geometry{
  top=1in,
  bottom=1in,
  left=1in,
  right=1in
}

\setlength{\parindent}{0pt}
\setlength{\parskip}{1em}

% --------------------
% Document
% --------------------
\begin{document}
\vspace*{-1.5cm}

% --------------------
% Header Section
% --------------------
\hspace{0.7cm}
\begin{minipage}[c]{0.7\textwidth}
\vspace{0.5cm}
\textbf{MTech CSE – 1st Semester} \\  
Student ID: \textbf{A125023} \\  
Student Name: \textbf{SRIJITA VERMA}
\end{minipage}
\hspace{0.5cm}
\begin{minipage}{0.2\textwidth}
\centering
\includegraphics[width=0.6\linewidth]{college_logo.jpg}
\end{minipage}

\vspace{1cm}

% --------------------
% Question Section
% --------------------
\section*{Question}

Is the \textsc{2-SAT} problem NP-hard? Can it be solved in polynomial time?
Explain your reasoning.

% --------------------
% Answer Section
% --------------------
\section*{Answer}
\section*{Short Answer}

The \textsc{2-SAT} problem is not NP-hard (unless $\mathbf{P}=\mathbf{NP}$).
Moreover, \textsc{2-SAT} can be solved in polynomial time, and in fact in linear
time, using graph-based algorithms such as implication graphs and strongly
connected components (SCCs).

\section*{What Is the 2-SAT Problem?}

The \textsc{2-SAT} (2-CNF-SAT) problem is a restricted version of the Boolean
satisfiability problem.

A Boolean formula is in \emph{2-CNF} if:
\begin{itemize}
  \item it is in conjunctive normal form (CNF), and
  \item each clause contains at most two literals.
\end{itemize}

For example:
\[
(x_1 \vee \neg x_2)
\;\wedge\;
(\neg x_1 \vee x_3)
\;\wedge\;
(x_2 \vee x_3).
\]

The problem asks whether there exists a truth assignment to the variables that
satisfies all clauses.

\section*{Is 2-SAT NP-Hard?}

\subsection*{Key Fact}

\textsc{2-SAT} is not NP-hard unless $\mathbf{P}=\mathbf{NP}$.

\subsection*{Reasoning}

The general \textsc{SAT} problem is NP-complete, and the restricted
\textsc{3-SAT} problem is also NP-complete. However, \textsc{2-SAT} is a strictly
easier special case.

If \textsc{2-SAT} were NP-hard, then since \textsc{2-SAT} can be solved in
polynomial time, every problem in $\mathbf{NP}$ would also be solvable in
polynomial time. This would imply:
\[
\mathbf{P} = \mathbf{NP},
\]
which is widely believed to be false. Hence, \textsc{2-SAT} is not NP-hard.

\section*{Polynomial-Time Solvability of 2-SAT}

\subsection*{Efficiency}

\textsc{2-SAT} can be solved in:
\[
O(n + m)
\]
time, where:
\begin{itemize}
  \item $n$ is the number of variables,
  \item $m$ is the number of clauses.
\end{itemize}

This efficiency is achieved using graph algorithms.

\section*{Implication Graph Formulation}

Each 2-CNF clause
\[
(a \vee b)
\]
is logically equivalent to:
\[
(\neg a \Rightarrow b) \;\wedge\; (\neg b \Rightarrow a).
\]

\subsection*{Graph Construction}

Construct a directed graph with:
\begin{itemize}
  \item one node for each literal $x$ and $\neg x$,
  \item for each clause $(a \vee b)$, add edges:
  \[
  \neg a \rightarrow b
  \quad \text{and} \quad
  \neg b \rightarrow a.
  \]
\end{itemize}

This graph is called the \emph{implication graph}.

\section*{Strongly Connected Components Criterion}

\subsection*{Key Theorem}

A 2-SAT formula is satisfiable if and only if, for no variable $x$, both $x$ and
$\neg x$ belong to the same strongly connected component of the implication
graph.

\subsection*{Explanation}

If $x \Rightarrow \neg x$ and $\neg x \Rightarrow x$, then assigning either truth
value to $x$ leads to a contradiction. Hence, no satisfying assignment exists.

\section*{Algorithm Outline}

\begin{itemize}
  \item Build the implication graph from the formula.
  \item Compute strongly connected components using Kosaraju’s or Tarjan’s
  algorithm.
  \item For each variable $x$:
  \begin{itemize}
    \item if $x$ and $\neg x$ lie in the same SCC, the formula is unsatisfiable;
    \item otherwise, the formula is satisfiable and a valid assignment can be
    constructed.
  \end{itemize}
\end{itemize}

All steps run in linear time.

\section*{Worked Example: Solving 2-SAT Using an Implication Graph}

Consider the 2-CNF formula:
\[
\varphi =
(x_1 \vee x_2)
\;\wedge\;
(\neg x_1 \vee x_3)
\;\wedge\;
(\neg x_2 \vee \neg x_3).
\]

We determine whether this formula is satisfiable using the implication graph
method.

\section*{Conversion of Clauses into Implications}

Each clause of the form $(a \vee b)$ is logically equivalent to:
\[
(\neg a \Rightarrow b) \;\wedge\; (\neg b \Rightarrow a).
\]

Applying this transformation to each clause:

\begin{itemize}
  \item $(x_1 \vee x_2)$ yields:
  \[
  \neg x_1 \Rightarrow x_2,
  \quad
  \neg x_2 \Rightarrow x_1.
  \]

  \item $(\neg x_1 \vee x_3)$ yields:
  \[
  x_1 \Rightarrow x_3,
  \quad
  \neg x_3 \Rightarrow \neg x_1.
  \]

  \item $(\neg x_2 \vee \neg x_3)$ yields:
  \[
  x_2 \Rightarrow \neg x_3,
  \quad
  x_3 \Rightarrow \neg x_2.
  \]
\end{itemize}

\section*{Construction of the Implication Graph}

The vertices of the implication graph correspond to the literals:
\[
\{\, x_1, \neg x_1, x_2, \neg x_2, x_3, \neg x_3 \,\}.
\]

The directed edges represent the implications derived above:
\[
\neg x_1 \rightarrow x_2,
\quad
\neg x_2 \rightarrow x_1,
\quad
x_1 \rightarrow x_3,
\quad
\neg x_3 \rightarrow \neg x_1,
\quad
x_2 \rightarrow \neg x_3,
\quad
x_3 \rightarrow \neg x_2.
\]

\section*{Analysis Using Strongly Connected Components}

We now compute the strongly connected components (SCCs) of the implication
graph.

For each variable $x_i$, we check whether the literals $x_i$ and $\neg x_i$
belong to the same SCC. If they do, the formula is unsatisfiable.

In this implication graph:
\begin{itemize}
  \item no variable and its negation appear in the same strongly connected
  component.
\end{itemize}

\section*{Conclusion for the Example}

Since no variable $x_i$ and its negation $\neg x_i$ lie in the same SCC, the
formula $\varphi$ is satisfiable.

A satisfying assignment can be constructed by assigning truth values according to
the topological order of the SCC condensation graph.
\section*{Why 2-SAT Is Easier Than 3-SAT}

The key difference lies in structure:
\begin{itemize}
  \item 2-SAT constraints form implications that can be analyzed using graph
  reachability.
  \item 3-SAT constraints create complex combinatorial interactions that cannot
  be captured by simple implication graphs.
\end{itemize}

Thus, \textsc{2-SAT} admits a global consistency check via SCCs, while
\textsc{3-SAT} remains NP-complete.

\section*{Complexity-Theoretic Classification}

\[
\begin{array}{|c|c|c|c|}
\hline
\textbf{Problem} & \textbf{In P} & \textbf{NP-Hard} & \textbf{NP-Complete} \\
\hline
\textsc{SAT} & \text{No} & \text{Yes} & \text{Yes} \\
\textsc{3-SAT} & \text{No} & \text{Yes} & \text{Yes} \\
\textsc{2-SAT} & \text{Yes} & \text{No} & \text{No} \\
\hline
\end{array}
\]

\section*{Final Conclusion}

\textsc{2-SAT} is not NP-hard (unless $\mathbf{P}=\mathbf{NP}$).  
It can be solved in polynomial time, and in fact in linear time, using implication
graphs and strongly connected components.

This makes \textsc{2-SAT} a classic example of how restricting problem structure
can dramatically reduce computational complexity.

\section*{Intuition}

2-SAT constraints behave like logical implications that must all be mutually
consistent. Graph reachability is sufficient to detect contradictions. In
contrast, 3-SAT allows richer interactions that require exponential search in
the worst case.

\end{document}
