\documentclass{article}

% --------------------
% Packages
% --------------------
\usepackage[utf8]{inputenc}
\usepackage{amsmath, amssymb}
\usepackage{graphicx}
\usepackage{geometry}

% --------------------
% Page Layout
% --------------------
\geometry{
  top=1in,
  bottom=1in,
  left=1in,
  right=1in
}

\setlength{\parindent}{0pt}
\setlength{\parskip}{1em}

% --------------------
% Document
% --------------------
\begin{document}
\vspace*{-1.5cm}

% --------------------
% Header Section
% --------------------
\hspace{0.7cm}
\begin{minipage}[c]{0.7\textwidth}
\vspace{0.5cm}
\textbf{MTech CSE – 1st Semester} \\  
Student ID: \textbf{A125023} \\  
Student Name: \textbf{SRIJITA VERMA}
\end{minipage}
\hspace{0.5cm}
\begin{minipage}{0.2\textwidth}
\centering
\includegraphics[width=0.6\linewidth]{college_logo.jpg}
\end{minipage}

\vspace{1cm}

% --------------------
% Question Section
% --------------------
\section*{Question}

Given a Boolean circuit instance whose output evaluates to \textsc{true},
explain how the correctness of the result can be verified in polynomial time
using Depth First Search (DFS).

% --------------------
% Answer Section
% --------------------
\section*{Answer}
\section*{Background: Boolean Circuits and Verification}

A Boolean circuit is a directed acyclic graph (DAG) where:
\begin{itemize}
  \item internal nodes are logic gates (\textsc{AND}, \textsc{OR}, \textsc{NOT}),
  \item leaves are input variables or constants,
  \item there is a single designated output gate.
\end{itemize}

A Boolean circuit instance is considered a \textsc{YES}-instance if, under a given
input assignment, the output gate evaluates to \textsc{true}.

The task here is verification, not computation. Given that the output is claimed
to be \textsc{true}, we must verify this claim efficiently.

\section*{Key Idea of Polynomial-Time Verification}

The correctness of a Boolean circuit’s output can be verified by:
\begin{itemize}
  \item traversing the circuit from the output gate to the input gates,
  \item recursively checking that each gate produces the correct output value,
  \item ensuring consistency with the logical semantics of each gate.
\end{itemize}

This process can be naturally implemented using Depth First Search (DFS).

\section*{Circuit as a Graph}

The Boolean circuit is modeled as a directed graph:
\begin{itemize}
  \item vertices represent gates or input literals,
  \item directed edges represent signal flow from inputs to outputs.
\end{itemize}

Since Boolean circuits are acyclic, the resulting graph is a DAG, which is ideal
for DFS traversal.

\section*{Verification Using DFS}

\subsection*{DFS Strategy}

A DFS is started from the output gate. The algorithm recursively verifies the
correctness of the output by checking its children.

At each visited node:
\begin{itemize}
  \item all input gates (children) are visited,
  \item their values are verified,
  \item the gate’s logical operation is checked against the claimed output.
\end{itemize}

\subsection*{Gate-by-Gate Verification Rules}

Suppose a gate claims to evaluate to \textsc{true}.

\begin{itemize}
  \item \textbf{AND gate:} All input gates must evaluate to \textsc{true}. DFS
  verifies each input recursively.
  \item \textbf{OR gate:} At least one input gate must evaluate to \textsc{true}.
  DFS verifies that at least one child evaluates to \textsc{true}.
  \item \textbf{NOT gate:} Its single input must evaluate to \textsc{false}. DFS
  verifies this condition.
  \item \textbf{Input literal:} The value is directly checked against the given
  input assignment.
\end{itemize}

If all checks succeed, the correctness of the output gate is verified.

\section*{DFS-Based Verification: Pseudocode}

Below is a high-level DFS procedure to verify that the output of a Boolean
circuit evaluates to \textsc{true} under a given input assignment.

\begin{verbatim}
Verify(gate):
    if gate is an input literal:
        return assigned truth value of gate

    if gate is AND:
        for each input child c:
            if Verify(c) == false:
                return false
        return true

    if gate is OR:
        for each input child c:
            if Verify(c) == true:
                return true
        return false

    if gate is NOT:
        let c be the single input child
        return NOT Verify(c)
\end{verbatim}

The verification starts by calling \texttt{Verify(output\_gate)}. If this call
returns \textsc{true}, the claimed output of the circuit is correct.
\section*{Why DFS Is Sufficient and Efficient}

\subsection*{Correctness}

DFS guarantees that:
\begin{itemize}
  \item every gate contributing to the output is visited,
  \item the logical correctness of each gate is verified bottom-up,
  \item no gate is processed before its dependencies.
\end{itemize}

Because the circuit is acyclic, DFS does not revisit nodes unnecessarily.

\subsection*{Time Complexity}

Let:
\begin{itemize}
  \item $n$ be the number of gates,
  \item $m$ be the number of wires (edges).
\end{itemize}

DFS visits each gate and each wire exactly once. Therefore, the verification runs
in:
\[
O(n + m),
\]
which is polynomial in the size of the circuit.

\section*{Relation to NP Verification}

The Boolean Circuit Value Problem (BCVP) is a canonical $\mathbf{NP}$-complete
problem.

This verification procedure demonstrates that:
\begin{itemize}
  \item given a certificate (the input assignment),
  \item the correctness of the output being \textsc{true}
\end{itemize}
can be verified in polynomial time.

This aligns precisely with the definition of $\mathbf{NP}$: \textsc{YES}-instances
have efficiently verifiable certificates. DFS provides a structured way to
perform this verification.

\section*{Relation to Circuit-SAT and NP-Completeness}

The \emph{Circuit-SAT} problem asks whether there exists an input assignment that
makes a given Boolean circuit evaluate to \textsc{true}. Circuit-SAT is a
canonical \textbf{NP-complete} problem.

In the context of NP, an input assignment serves as a \emph{certificate} for a
YES-instance of Circuit-SAT.

The DFS-based verification described above shows that:
\begin{itemize}
  \item given a candidate assignment,
  \item the correctness of the circuit’s output being \textsc{true},
  \item can be verified in polynomial time.
\end{itemize}

This directly satisfies the definition of NP: YES-instances admit polynomial-time
verifiable certificates.

Thus, DFS provides a concrete and efficient mechanism for verifying certificates
in NP-complete problems such as Circuit-SAT.
\section*{Why BFS Is Not Necessary}

Although Breadth First Search could traverse the circuit, DFS is preferable
because:
\begin{itemize}
  \item it naturally mirrors recursive gate evaluation,
  \item it supports bottom-up verification,
  \item it requires minimal auxiliary memory via recursion or a stack.
\end{itemize}

Thus, DFS is both conceptually and practically well-suited for this task.

\section*{Final Conclusion}

The correctness of a Boolean circuit’s \textsc{true} output can be verified in
polynomial time using DFS.

By performing a depth-first traversal from the output gate and verifying the
logical consistency of each gate with its inputs, correctness is ensured in
linear time with respect to the circuit size.

\section*{Intuition}

Evaluating a Boolean circuit is like checking a proof tree:
\begin{itemize}
  \item DFS follows the reasoning backward,
  \item verifies each step,
  \item and confirms that the final conclusion (\textsc{true} output) is justified.
\end{itemize}

\end{document}
