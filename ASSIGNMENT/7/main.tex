\documentclass{article}

% --------------------
% Packages
% --------------------
\usepackage[utf8]{inputenc}
\usepackage{amsmath, amssymb}
\usepackage{graphicx}
\usepackage{geometry}

% --------------------
% Page Layout
% --------------------
\geometry{
  top=1in,
  bottom=1in,
  left=1in,
  right=1in
}

\setlength{\parindent}{0pt}
\setlength{\parskip}{1em}

% --------------------
% Document
% --------------------
\begin{document}
\vspace*{-1.5cm}

% --------------------
% Header Section
% --------------------
\hspace{0.7cm}
\begin{minipage}[c]{0.7\textwidth}
\vspace{0.5cm}
\textbf{MTech CSE – 1st Semester} \\  
Student ID: \textbf{A125023} \\  
Student Name: \textbf{SRIJITA VERMA}
\end{minipage}
\hspace{0.5cm}
\begin{minipage}{0.2\textwidth}
\centering
\includegraphics[width=0.6\linewidth]{college_logo.jpg}
\end{minipage}

\vspace{1cm}

% --------------------
% Question Section
% --------------------
\section*{Question}
Prove that positive-definite matrices are suitable for LU decomposition and do
not require pivoting to avoid division by zero in the recursive strategy.

% --------------------
% Answer Section
% --------------------
\section*{Answer}
\section*{Overview and Strategy}

To answer this question, we must show two things:
\begin{itemize}
  \item Existence of LU decomposition without pivoting for positive-definite
        matrices.
  \item Absence of zero pivots during Gaussian elimination, so no division by
        zero occurs.
\end{itemize}

The key idea is that positive-definite matrices have strictly positive leading
principal minors, which guarantees that all pivots encountered in LU
decomposition are non-zero and positive.

\section*{Preliminaries and Definitions}

\subsection*{Definition: Positive-Definite Matrix}

A real symmetric matrix $A \in \mathbb{R}^{n \times n}$ is positive-definite if
\[
x^{T} A x > 0 \quad \text{for all non-zero } x \in \mathbb{R}^{n}.
\]

\subsection*{Definition: LU Decomposition (Without Pivoting)}

An LU decomposition of a matrix $A$ is a factorization
\[
A = LU,
\]
where:
\begin{itemize}
  \item $L$ is a unit lower triangular matrix,
  \item $U$ is an upper triangular matrix.
\end{itemize}

In Gaussian elimination, the diagonal entries of $U$ are the pivot elements.
Pivoting is required only if a pivot becomes zero (or numerically unstable).

\section*{Key Theoretical Result}

\subsection*{Theorem (Sylvester’s Criterion)}

A real symmetric matrix $A$ is positive-definite if and only if all its leading
principal minors are strictly positive, that is,
\[
\det(A_k) > 0 \quad \text{for } k = 1, 2, \ldots, n,
\]
where $A_k$ denotes the $k \times k$ leading principal submatrix of $A$.

\section*{Step 1: Connection Between LU Decomposition and Leading Principal Minors}

For an LU decomposition without pivoting, the pivots satisfy:
\[
u_{kk} = \frac{\det(A_k)}{\det(A_{k-1})},
\quad k = 1, 2, \ldots, n,
\]
with the convention $\det(A_0) = 1$.

Thus, a zero pivot $u_{kk} = 0$ occurs if and only if $\det(A_k) = 0$.
Therefore, non-zero leading principal minors guarantee non-zero pivots.

\section*{Step 2: Apply Positive-Definiteness}

Since $A$ is positive-definite:
\begin{itemize}
  \item $A$ is symmetric,
  \item by Sylvester’s criterion, $\det(A_k) > 0$ for all $k$.
\end{itemize}

Hence,
\[
u_{kk} = \frac{\det(A_k)}{\det(A_{k-1})} > 0
\quad \text{for all } k.
\]

This shows that:
\begin{itemize}
  \item all pivots are strictly positive,
  \item no division by zero can occur during LU decomposition.
\end{itemize}

\section*{Step 3: Suitability for the Recursive LU Strategy}

The recursive LU algorithm computes:
\[
u_{kk} = a_{kk}^{(k)}, \quad
l_{ik} = \frac{a_{ik}^{(k)}}{u_{kk}}.
\]

Since $u_{kk} > 0$ for all $k$:
\begin{itemize}
  \item each division is well-defined,
  \item the recursive elimination proceeds safely without pivoting.
\end{itemize}

\section*{Step 4: Why Pivoting Is Unnecessary}

Pivoting is used to:
\begin{itemize}
  \item avoid division by zero,
  \item improve numerical stability.
\end{itemize}

For positive-definite matrices:
\begin{itemize}
  \item division by zero cannot occur,
  \item pivots are guaranteed to be positive,
  \item the matrix is well-conditioned in theory.
\end{itemize}

Thus, pivoting is unnecessary to ensure correctness.

\section*{Relation to Numerical Stability}

In numerical linear algebra, pivoting is often introduced not only to avoid
division by zero but also to improve numerical stability by controlling the
growth of rounding errors.

For positive-definite matrices, the pivots in LU decomposition are guaranteed
to be strictly positive and bounded away from zero. As a result:
\begin{itemize}
  \item division by zero cannot occur,
  \item large element growth is avoided,
  \item the amplification of rounding errors is limited.
\end{itemize}

Consequently, LU decomposition without pivoting is numerically stable for
positive-definite matrices in exact arithmetic and performs reliably in
floating-point computations.
\section*{Worked Numerical Example}

Consider the symmetric matrix:
\[
A =
\begin{bmatrix}
4 & 2 & 2 \\
2 & 5 & 1 \\
2 & 1 & 3
\end{bmatrix}
\]

We first verify that $A$ is positive-definite. All leading principal minors are:
\[
\det(A_1) = 4 > 0,
\quad
\det(A_2) =
\begin{vmatrix}
4 & 2 \\
2 & 5
\end{vmatrix}
= 16 > 0,
\quad
\det(A_3) = 36 > 0.
\]

Hence, $A$ is positive-definite.

\subsection*{LU Decomposition Without Pivoting}

Applying Gaussian elimination without pivoting:

\[
L =
\begin{bmatrix}
1 & 0 & 0 \\
\frac{1}{2} & 1 & 0 \\
\frac{1}{2} & 0 & 1
\end{bmatrix},
\quad
U =
\begin{bmatrix}
4 & 2 & 2 \\
0 & 4 & 0 \\
0 & 0 & 2
\end{bmatrix}
\]

All pivots ($4$, $4$, and $2$) are strictly positive, and no division by zero
occurs. This confirms that LU decomposition proceeds safely without pivoting.
\section*{Additional Insight: Relation to Cholesky Decomposition}

Every positive-definite matrix admits a Cholesky decomposition:
\[
A = LL^{T},
\]
which is a special case of LU decomposition with
\[
U = L^{T}.
\]

In this case:
\begin{itemize}
  \item all pivots are square roots of positive numbers,
  \item no pivoting is required.
\end{itemize}

This further reinforces that positive-definite matrices are inherently suitable
for triangular factorizations.

\section*{Comparison: LU vs Cholesky Decomposition}

Both LU and Cholesky decompositions can be used for positive-definite matrices,
but Cholesky decomposition is more specialized.

\begin{itemize}
  \item \textbf{LU Decomposition} applies to a broader class of matrices and
        factors $A$ as $LU$.
  \item \textbf{Cholesky Decomposition} applies only to symmetric
        positive-definite matrices and factors $A$ as $LL^{T}$.
\end{itemize}

From a computational perspective:
\begin{itemize}
  \item Cholesky decomposition requires roughly half the number of arithmetic
        operations compared to LU decomposition.
  \item Cholesky is inherently stable and does not require pivoting.
  \item LU decomposition without pivoting is safe for positive-definite
        matrices but is less efficient than Cholesky.
\end{itemize}

Therefore, while positive-definite matrices admit LU decomposition without
pivoting, Cholesky decomposition is generally preferred in practice.
\section*{Final Conclusion}

Positive-definite matrices admit LU decomposition without pivoting because:
\begin{itemize}
  \item all leading principal minors are strictly positive,
  \item all pivots in Gaussian elimination are non-zero,
  \item division by zero cannot occur in the recursive strategy.
\end{itemize}

\section*{Interpretation (Intuition)}

Positive-definite matrices represent strictly convex quadratic forms. Eliminating
variables never collapses dimensionality, and each step of Gaussian elimination
preserves positivity. Hence, the algorithm proceeds safely without row exchanges.

\subsection*{Remark}

Positive-definite matrices arising from physical or optimization problems are
often well-conditioned, which further enhances the numerical reliability of
direct factorization methods.
\end{document}
