\documentclass{article}

% --------------------
% Packages
% --------------------
\usepackage[utf8]{inputenc}
\usepackage{amsmath, amssymb}
\usepackage{graphicx}
\usepackage{geometry}

% --------------------
% Page Layout
% --------------------
\geometry{
  top=1in,
  bottom=1in,
  left=1in,
  right=1in
}

\setlength{\parindent}{0pt}
\setlength{\parskip}{1em}

% --------------------
% Document
% --------------------
\begin{document}
\vspace*{-1.5cm}

% --------------------
% Header Section
% --------------------
\hspace{0.7cm}
\begin{minipage}[c]{0.7\textwidth}
\vspace{0.5cm}
\textbf{MTech CSE – 1st Semester} \\  
Student ID: \textbf{A125023} \\  
Student Name: \textbf{SRIJITA VERMA}
\end{minipage}
\hspace{0.5cm}
\begin{minipage}{0.2\textwidth}
\centering
\includegraphics[width=0.6\linewidth]{college_logo.jpg}
\end{minipage}

\vspace{1cm}

% --------------------
% Question Section
% --------------------
\section*{Question}
Prove that if a matrix $A$ is non-singular, then its Schur complement is also
non-singular.

% --------------------
% Answer Section
% --------------------
\section*{Answer}
\section*{Preliminaries and Definitions}

Let $A \in \mathbb{R}^{n \times n}$ be a square matrix partitioned as
\[
A =
\begin{bmatrix}
A_{11} & A_{12} \\
A_{21} & A_{22}
\end{bmatrix},
\]
where:
\begin{itemize}
  \item $A_{11} \in \mathbb{R}^{k \times k}$,
  \item $A_{22} \in \mathbb{R}^{(n-k) \times (n-k)}$,
\end{itemize}
and the block dimensions are compatible.

We assume that $A_{11}$ is invertible.

This assumption ensures that the Schur complement of $A_{11}$ in $A$ is
well-defined.

\subsection*{Definition: Schur Complement}

The Schur complement of $A_{11}$ in $A$ is defined as
\[
S = A_{22} - A_{21} A_{11}^{-1} A_{12}.
\]

\section*{Goal}

We are given that:
\begin{itemize}
  \item the full matrix $A$ is non-singular,
\end{itemize}
and we must prove that:
\begin{itemize}
  \item the Schur complement $S$ is also non-singular.
\end{itemize}

\section*{Key Idea of the Proof}

The proof relies on:
\begin{itemize}
  \item block Gaussian elimination,
  \item factorization of the matrix $A$,
  \item the fact that a product of matrices is invertible if and only if each
        factor is invertible.
\end{itemize}

\section*{Step 1: Block Factorization of $A$}

Using block Gaussian elimination, the matrix $A$ can be factored as:
\[
A =
\begin{bmatrix}
I & 0 \\
A_{21} A_{11}^{-1} & I
\end{bmatrix}
\begin{bmatrix}
A_{11} & A_{12} \\
0 & S
\end{bmatrix},
\]
where
\[
S = A_{22} - A_{21} A_{11}^{-1} A_{12}.
\]

\section*{Step 2: Analyze Invertibility of Each Factor}

\subsection*{First Factor}

Let
\[
L =
\begin{bmatrix}
I & 0 \\
A_{21} A_{11}^{-1} & I
\end{bmatrix}.
\]

This is a block lower triangular matrix with identity matrices on the diagonal.
Hence, $L$ is invertible.

\subsection*{Second Factor}

Let
\[
U =
\begin{bmatrix}
A_{11} & A_{12} \\
0 & S
\end{bmatrix}.
\]

This is a block upper triangular matrix. Its determinant is given by
\[
\det(U) = \det(A_{11}) \cdot \det(S).
\]

\section*{Step 3: Use Non-Singularity of $A$}

Since
\[
A = LU,
\]
and $A$ is non-singular while $L$ is invertible, it follows that $U$ must also
be non-singular. Therefore,
\[
\det(A_{11}) \cdot \det(S) \neq 0.
\]

Because $A_{11}$ is assumed invertible, we have
\[
\det(A_{11}) \neq 0,
\]
which implies
\[
\det(S) \neq 0.
\]

Thus, the Schur complement $S$ is non-singular.

\section*{Final Conclusion}

If $A$ is non-singular and $A_{11}$ is invertible, then its Schur complement
\[
S = A_{22} - A_{21} A_{11}^{-1} A_{12}
\]
is also non-singular.

\section*{Worked Numerical Example}

Consider the matrix:
\[
A =
\begin{bmatrix}
2 & 1 & 0 \\
1 & 3 & 1 \\
0 & 1 & 2
\end{bmatrix}
\]

Partition $A$ as:
\[
A =
\begin{bmatrix}
A_{11} & A_{12} \\
A_{21} & A_{22}
\end{bmatrix}
\]
where:
\[
A_{11} = [2], \quad
A_{12} = [1 \;\; 0], \quad
A_{21} =
\begin{bmatrix}
1 \\
0
\end{bmatrix}, \quad
A_{22} =
\begin{bmatrix}
3 & 1 \\
1 & 2
\end{bmatrix}
\]

Since $\det(A) = 8 \neq 0$, the matrix $A$ is non-singular, and
$A_{11}$ is invertible with $A_{11}^{-1} = \frac{1}{2}$.

The Schur complement of $A_{11}$ is:
\[
S = A_{22} - A_{21} A_{11}^{-1} A_{12}
=
\begin{bmatrix}
3 & 1 \\
1 & 2
\end{bmatrix}
-
\begin{bmatrix}
1 \\
0
\end{bmatrix}
\frac{1}{2}
\begin{bmatrix}
1 & 0
\end{bmatrix}
\]

\[
S =
\begin{bmatrix}
\frac{5}{2} & 1 \\
1 & 2
\end{bmatrix}
\]

The determinant of $S$ is:
\[
\det(S) = \frac{5}{2} \cdot 2 - 1 = 4 \neq 0
\]

Thus, the Schur complement is non-singular, illustrating the result.
\section*{Converse Result}

We now prove the converse statement.

\subsection*{Theorem}
If $A_{11}$ and its Schur complement
\[
S = A_{22} - A_{21} A_{11}^{-1} A_{12}
\]
are both non-singular, then the block matrix
\[
A =
\begin{bmatrix}
A_{11} & A_{12} \\
A_{21} & A_{22}
\end{bmatrix}
\]
is non-singular.

\subsection*{Proof}

Using block matrix factorization, we write:
\[
A =
\begin{bmatrix}
I & 0 \\
A_{21} A_{11}^{-1} & I
\end{bmatrix}
\begin{bmatrix}
A_{11} & A_{12} \\
0 & S
\end{bmatrix}
\]

The first factor is invertible since it is block lower triangular with
identity matrices on the diagonal. The second factor is invertible
because both $A_{11}$ and $S$ are invertible.

Since $A$ is a product of two invertible matrices, it is itself
invertible. Hence, $A$ is non-singular.
\hfill $\square$
\section*{Relation to LU and LUP Decomposition}

The Schur complement arises naturally during LU and LUP decomposition
of block matrices.

In block LU decomposition, a matrix $A$ is factorized as:
\[
A =
\begin{bmatrix}
A_{11} & A_{12} \\
A_{21} & A_{22}
\end{bmatrix}
=
\begin{bmatrix}
I & 0 \\
A_{21} A_{11}^{-1} & I
\end{bmatrix}
\begin{bmatrix}
A_{11} & A_{12} \\
0 & S
\end{bmatrix}
\]

where $S$ is the Schur complement of $A_{11}$.

Thus, the Schur complement represents the remaining block after
eliminating the variables corresponding to $A_{11}$ using Gaussian
elimination. Non-singularity of $S$ ensures that the elimination
process can proceed without breakdown.

In the presence of pivoting, the same structure appears in LUP
decomposition, where row permutations are applied before computing
the Schur complement. The non-singularity of Schur complements is
therefore essential for the stability and correctness of LU/LUP-based
algorithms. This block factorization underlies the $O(n^3)$ complexity of LU decomposition
and the correctness of subsequent $O(n^2)$ solve phases.

\section*{Remarks}

This result is fundamental in:
\begin{itemize}
  \item block LU decomposition,
  \item numerical linear algebra,
  \item matrix inversion formulas,
  \item optimization and control theory.
\end{itemize}

A symmetric result holds if $A_{22}$ is invertible, in which case the Schur
complement of $A_{22}$ is also non-singular.

\section*{Interpretation (Intuition)}

Schur complements arise naturally when eliminating variables in block linear
systems. Non-singularity of the full system implies that no information is lost
during elimination. Hence, the reduced system represented by the Schur
complement must also be non-singular.

\end{document}
