\documentclass{article}

% --------------------
% Packages
% --------------------
\usepackage[utf8]{inputenc}
\usepackage{amsmath, amssymb}
\usepackage{graphicx}
\usepackage{geometry}

% --------------------
% Page Layout
% --------------------
\geometry{
  top=1in,
  bottom=1in,
  left=1in,
  right=1in
}

\setlength{\parindent}{0pt}
\setlength{\parskip}{1em}

% --------------------
% Document
% --------------------
\begin{document}
\vspace*{-1.5cm}

% --------------------
% Header Section
% --------------------
\hspace{0.7cm}
\begin{minipage}[c]{0.7\textwidth}
\vspace{0.5cm}
\textbf{MTech CSE – 1st Semester} \\  
Student ID: \textbf{A125023} \\  
Student Name: \textbf{SRIJITA VERMA}
\end{minipage}
\hspace{0.5cm}
\begin{minipage}{0.2\textwidth}
\centering
\includegraphics[width=0.6\linewidth]{college_logo.jpg}
\end{minipage}

\vspace{1cm}

% --------------------
% Question Section
% --------------------
\section*{Question}

Solve the following recurrence relation arising from the LUP decomposition
solve procedure:
\[
T(n) = \sum_{i=1}^{n} \left[ O(1) + \sum_{j=1}^{i-1} O(1) \right]
       + \sum_{i=1}^{n} \left[ O(1) + \sum_{j=i+1}^{n} O(1) \right].
\]

% --------------------
% Answer Section
% --------------------
\section*{Answer}
\section*{Step 1: Interpret the Recurrence}

The recurrence consists of two main summations, each iterating over
$i = 1$ to $n$.

Each part models the cost of:
\begin{itemize}
  \item constant-time operations, and
  \item inner loops that perform a constant-time operation multiple times.
\end{itemize}

Our goal is to count the total number of constant-time operations and determine
the overall asymptotic complexity.

\section*{Step 2: Simplify the First Summation}

Consider the first term:
\[
\sum_{i=1}^{n} \left[ O(1) + \sum_{j=1}^{i-1} O(1) \right].
\]

\subsection*{Inner Summation}

The inner sum
\[
\sum_{j=1}^{i-1} O(1)
\]
executes $i-1$ times. Hence:
\[
\sum_{j=1}^{i-1} O(1) = O(i).
\]

\subsection*{First Summation Becomes}

\[
\sum_{i=1}^{n} \left[ O(1) + O(i) \right]
= \sum_{i=1}^{n} O(i).
\]

\section*{Step 3: Evaluate the First Summation}

We know that:
\[
\sum_{i=1}^{n} i = \frac{n(n+1)}{2}.
\]

Therefore:
\[
\sum_{i=1}^{n} O(i) = O(n^2).
\]

\section*{Step 4: Simplify the Second Summation}

Now consider the second term:
\[
\sum_{i=1}^{n} \left[ O(1) + \sum_{j=i+1}^{n} O(1) \right].
\]

\subsection*{Inner Summation}

The inner sum
\[
\sum_{j=i+1}^{n} O(1)
\]
executes $n-i$ times. Hence:
\[
\sum_{j=i+1}^{n} O(1) = O(n-i).
\]

\subsection*{Second Summation Becomes}

\[
\sum_{i=1}^{n} \left[ O(1) + O(n-i) \right]
= \sum_{i=1}^{n} O(n-i).
\]

\section*{Step 5: Evaluate the Second Summation}

As $i$ ranges from $1$ to $n$, the quantity $n-i$ ranges from $n-1$ down to $0$.
Thus:
\[
\sum_{i=1}^{n} (n-i) = \sum_{k=0}^{n-1} k = \frac{n(n-1)}{2}.
\]

Therefore:
\[
\sum_{i=1}^{n} O(n-i) = O(n^2).
\]

\section*{Step 6: Combine Both Parts}

From the above steps:
\begin{itemize}
  \item The first summation contributes $O(n^2)$,
  \item The second summation contributes $O(n^2)$.
\end{itemize}

Hence:
\[
T(n) = O(n^2) + O(n^2) = O(n^2).
\]

\section*{Final Answer}

\[
\boxed{T(n) = O(n^2)}
\]

\paragraph{Comparison with LUP Decomposition Cost.}
While the LUP decomposition itself requires $O(n^3)$ time, the solve phase
analyzed here runs in $O(n^2)$ time, making it efficient when solving linear
systems with multiple right-hand sides once the factorization has been computed.

\section*{Relation to LUP Solve Algorithm (Pseudocode)}

The recurrence corresponds to the solve phase of the LUP decomposition, which
consists of forward and backward substitution.

\subsection*{Forward Substitution (Solving $Ly = Pb$)}

\begin{verbatim}
for i = 1 to n:
    y[i] = b[i]
    for j = 1 to i-1:
        y[i] = y[i] - L[i][j] * y[j]
\end{verbatim}

The outer loop runs $n$ times, and the inner loop executes $i-1$ constant-time
operations, giving rise to the summation
\[
\sum_{i=1}^{n} \left[ O(1) + \sum_{j=1}^{i-1} O(1) \right].
\]

\subsection*{Backward Substitution (Solving $Ux = y$)}

\begin{verbatim}
for i = n down to 1:
    x[i] = y[i]
    for j = i+1 to n:
        x[i] = x[i] - U[i][j] * x[j]
\end{verbatim}

Here, the inner loop executes $n-i$ constant-time operations, producing the
summation
\[
\sum_{i=1}^{n} \left[ O(1) + \sum_{j=i+1}^{n} O(1) \right].
\]

Together, these two procedures account for the full recurrence analyzed above.
\section*{Relation to LUP Solve Algorithm (Pseudocode Interpretation)}
The recurrence solved earlier corresponds to the \emph{solve phase} after LUP
decomposition, not to the decomposition itself. This solve phase consists of two
steps:
\begin{itemize}
  \item Forward substitution to solve $Ly = Pb$,
  \item Backward substitution to solve $Ux = y$.
\end{itemize}

Below, we present the pseudocode-level interpretation and show how each loop
maps directly to the recurrence relation.

\section*{Forward Substitution (Lower Triangular Solve)}

\subsection*{Pseudocode}

\begin{verbatim}
for i = 1 to n:
    y[i] = b[i]
    for j = 1 to i-1:
        y[i] = y[i] - L[i][j] * y[j]
\end{verbatim}

\subsection*{Cost Analysis}

\begin{itemize}
  \item The outer loop runs $n$ times.
  \item The inner loop runs $i-1$ times for a given $i$.
  \item Each inner operation takes $O(1)$ time.
\end{itemize}

Thus, the total cost of forward substitution is:
\[
\sum_{i=1}^{n} \left[ O(1) + \sum_{j=1}^{i-1} O(1) \right],
\]
which matches the first summation in the recurrence.

\section*{Backward Substitution (Upper Triangular Solve)}

\subsection*{Pseudocode}

\begin{verbatim}
for i = n down to 1:
    x[i] = y[i]
    for j = i+1 to n:
        x[i] = x[i] - U[i][j] * x[j]
\end{verbatim}

\subsection*{Cost Analysis}

\begin{itemize}
  \item The outer loop runs $n$ times.
  \item The inner loop runs $n-i$ times for a given $i$.
  \item Each inner operation takes $O(1)$ time.
\end{itemize}

Thus, the total cost of backward substitution is:
\[
\sum_{i=1}^{n} \left[ O(1) + \sum_{j=i+1}^{n} O(1) \right],
\]
which matches the second summation in the recurrence.

\section*{Combined Interpretation}

Combining both phases, the total running time of the LUP solve procedure is:
\[
T(n)
=
\sum_{i=1}^{n} \left[ O(1) + \sum_{j=1}^{i-1} O(1) \right]
+
\sum_{i=1}^{n} \left[ O(1) + \sum_{j=i+1}^{n} O(1) \right].
\]

This recurrence therefore represents exactly the combined cost of:
\begin{itemize}
  \item Forward substitution, and
  \item Backward substitution.
\end{itemize}

As shown earlier, this simplifies to:
\[
T(n) = O(n^2).
\]

\section*{Result}

The recurrence arises directly from the nested-loop structure of forward and
backward substitution in the LUP solve procedure. This confirms that, while the
LUP decomposition itself requires $O(n^3)$ time, each subsequent solve step
runs in quadratic time.

\paragraph{Practical Efficiency.}
Both forward and backward substitution access matrix entries in a largely
sequential manner, which leads to good cache locality in practice. This memory
access pattern further contributes to their efficiency beyond the quadratic
asymptotic time complexity.

\section*{Interpretation in the Context of LUP Decomposition}

This recurrence corresponds to the forward substitution and backward substitution
steps used when solving:
\[
Ax = b \quad \text{using} \quad PA = LU.
\]

The key insight is:
\begin{itemize}
  \item LUP decomposition itself takes $O(n^3)$ time,
  \item Once $L$ and $U$ are available, each solve step (forward and backward
        substitution) takes $O(n^2)$ time.
\end{itemize}

This confirms that the solve phase is quadratic, not cubic.

\section*{Conclusion}

The given recurrence simplifies to a quadratic time complexity. This aligns with
the theoretical expectation for forward and backward substitution in LUP-based
linear system solvers, where nested loops execute a linear number of constant-time
operations.

\end{document}
