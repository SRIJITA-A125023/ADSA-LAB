\documentclass{article}

% --------------------
% Packages
% --------------------
\usepackage[utf8]{inputenc}
\usepackage{amsmath, amssymb}
\usepackage{graphicx}
\usepackage{geometry}

% --------------------
% Page Layout
% --------------------
\geometry{
  top=1in,
  bottom=1in,
  left=1in,
  right=1in
}

\setlength{\parindent}{0pt}
\setlength{\parskip}{1em}

% --------------------
% Document
% --------------------
\begin{document}
\vspace*{-1.5cm}

% --------------------
% Header Section
% --------------------
\hspace{0.7cm}
\begin{minipage}[c]{0.7\textwidth}
\vspace{0.5cm}
\textbf{MTech CSE – 1st Semester} \\  
Student ID: \textbf{A125023} \\  
Student Name: \textbf{SRIJITA VERMA}
\end{minipage}
\hspace{0.5cm}
\begin{minipage}{0.2\textwidth}
\centering
\includegraphics[width=0.6\linewidth]{college_logo.jpg}
\end{minipage}

\vspace{1cm}

% --------------------
% Question Section
% --------------------
\section*{Question}
Show that in any heap containing $n$ elements, the number of nodes at height $h$
is at most:
\[
\left\lceil \frac{n}{2^{h+1}} \right\rceil
\]

% --------------------
% Answer Section
% --------------------
\section*{Answer}

\section*{Assumptions and Model}

We assume a binary heap, as defined in standard algorithm texts (CLRS-style), with
the following properties:
\begin{itemize}
  \item The heap is stored as a complete binary tree.
  \item The heap-order property (min-heap or max-heap) is irrelevant for this
        problem; only the tree structure matters.
\end{itemize}

Note that the heap-order property (min-heap or max-heap) plays no role in this argument; the proof depends solely on the structural completeness of the heap.

\section*{Definitions}

\subsection*{Definition 1: Complete Binary Tree}

A binary tree is complete if:
\begin{itemize}
  \item Every level except possibly the last is completely filled.
  \item Nodes in the last level are filled from left to right.
\end{itemize}

A heap always satisfies this property.

\subsection*{Definition 2: Height of a Node}

The height of a node is defined as the number of edges on the longest downward
path from that node to a leaf.

Thus:
\begin{itemize}
  \item Leaves have height $0$.
  \item A node whose children are leaves has height $1$.
  \item The root has the maximum height in the heap.
\end{itemize}

\section*{Goal Restated Precisely}

Let:
\begin{itemize}
  \item $n$ be the total number of nodes in the heap,
  \item $h \ge 0$ be a fixed integer,
  \item $N_h$ be the number of nodes whose height is exactly $h$.
\end{itemize}

We must prove:
\[
N_h \le \left\lceil \frac{n}{2^{h+1}} \right\rceil.
\]

\section*{Core Insight of the Proof}

The proof relies on counting arguments based on subtree sizes.

A node of height $h$ must have a sufficiently large subtree below it. Because the
heap is complete, such subtrees cannot be arbitrarily small.

\section*{Step 1: Minimum Size of a Subtree of Height $h$}

Consider any node $v$ of height $h$.

By definition, the longest downward path from $v$ to a leaf has length $h$.
Therefore, the subtree rooted at $v$ has at least $h+1$ levels.

\subsection*{Lemma 1}

The minimum number of nodes in a binary tree of height $h$ is:
\[
2^{h+1} - 1.
\]

For the purpose of deriving an upper bound, we may use the weaker inequality that
each such subtree contains at least $2^{h+1}$ nodes. Replacing
$2^{h+1} - 1$ by $2^{h+1}$ simplifies the counting argument without affecting the
correctness of the bound, since the resulting inequality remains valid.

\subsection*{Justification}

The smallest tree of height $h$ is a perfect binary tree. Such a tree has:
\begin{itemize}
  \item $1$ node at level $0$,
  \item $2$ nodes at level $1$,
  \item $2^h$ nodes at level $h$.
\end{itemize}

Thus, the total number of nodes is:
\[
1 + 2 + 4 + \cdots + 2^h = 2^{h+1} - 1.
\]

\subsection*{Simplification for Counting}

Since:
\[
2^{h+1} - 1 \ge 2^{h+1}/2,
\]
we use the weaker but sufficient bound:
\[
\text{Subtree size} \ge 2^{h+1}.
\]

This simplifies the algebra while preserving correctness.

\section*{Step 2: Disjointness of Subtrees}

Consider all nodes of height exactly $h$.

\subsection*{Key Observation}

No node of height $h$ can be an ancestor of another node of height $h$.

Therefore, the subtrees rooted at nodes of height $h$ are pairwise disjoint. No
node in the heap belongs to more than one such subtree.

\section*{Step 3: Global Counting Argument}

Let $N_h$ be the number of nodes at height $h$.

Each such node roots a subtree containing at least $2^{h+1}$ nodes. Hence, the
total number of nodes covered by these subtrees is at least:
\[
N_h \cdot 2^{h+1}.
\]

Since the heap contains only $n$ nodes in total:
\[
N_h \cdot 2^{h+1} \le n.
\]

\section*{Step 4: Solving the Inequality}

Dividing both sides by $2^{h+1}$:
\[
N_h \le \frac{n}{2^{h+1}}.
\]

Since $N_h$ must be an integer, we take the ceiling:
\[
N_h \le \left\lceil \frac{n}{2^{h+1}} \right\rceil.
\]

\section*{Illustrative Example}

For example, in a heap with $n = 15$ nodes, there can be at most
$\left\lceil \dfrac{15}{2^{h+1}} \right\rceil$ nodes of height $h$.
For $h = 2$, this bound gives at most $2$ nodes, which is consistent
with the structure of a complete binary heap.


\section*{Final Result}

In a heap with $n$ elements, the number of nodes of height $h$ is at most:
\[
\left\lceil \frac{n}{2^{h+1}} \right\rceil.
\]

As a consistency check, when $h = 0$, the bound yields at most
$\left\lceil \dfrac{n}{2} \right\rceil$ nodes, which matches the fact that in a
complete binary tree, at most half of the nodes can be leaves.


\section*{Intuition (Conceptual Explanation)}

\begin{itemize}
  \item Nodes closer to the root have larger height.
  \item Larger height implies larger required subtrees.
  \item Since the total number of nodes is fixed, only a few nodes can have large
        height.
\end{itemize}

Therefore:
\begin{itemize}
  \item Most nodes in a heap are close to the leaves.
  \item Very few nodes are near the root.
\end{itemize}

\section*{Importance of This Result}

This bound is fundamental in algorithm analysis, especially for:
\begin{itemize}
  \item \textsc{Build-Heap},
  \item \textsc{Heapify},
  \item proving that \textsc{Build-Heap} runs in $O(n)$ time.
\end{itemize}

It allows us to weight the cost of Heapify by the number of nodes at each height.

\subsection*{Observation}

This bound captures the intuition that heaps contain many nodes near the leaves
and very few near the root. This structural property plays a crucial role in the
linear-time analysis of the \textsc{Build-Heap} algorithm.
\end{document}
