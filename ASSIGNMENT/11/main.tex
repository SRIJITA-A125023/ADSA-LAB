\documentclass{article}

% --------------------
% Packages
% --------------------
\usepackage[utf8]{inputenc}
\usepackage{amsmath, amssymb}
\usepackage{graphicx}
\usepackage{geometry}

% --------------------
% Page Layout
% --------------------
\geometry{
  top=1in,
  bottom=1in,
  left=1in,
  right=1in
}

\setlength{\parindent}{0pt}
\setlength{\parskip}{1em}

% --------------------
% Document
% --------------------
\begin{document}
\vspace*{-1.5cm}

% --------------------
% Header Section
% --------------------
\hspace{0.7cm}
\begin{minipage}[c]{0.7\textwidth}
\vspace{0.5cm}
\textbf{MTech CSE – 1st Semester} \\  
Student ID: \textbf{A125023} \\  
Student Name: \textbf{SRIJITA VERMA}
\end{minipage}
\hspace{0.5cm}
\begin{minipage}{0.2\textwidth}
\centering
\includegraphics[width=0.6\linewidth]{college_logo.jpg}
\end{minipage}

\vspace{1cm}

% --------------------
% Question Section
% --------------------
\section*{Question}
Define the class \textbf{Co-NP}. Explain the type of problems that belong to this
complexity class.

% --------------------
% Answer Section
% --------------------
\section*{Answer}
\section*{Preliminaries: Decision Problems and NP}

In computational complexity theory, we primarily study \emph{decision problems},
whose answers are either \textsc{YES} or \textsc{NO}.

Recall that a decision problem $L$ belongs to the class $\mathbf{NP}$ if:
\begin{itemize}
  \item for every input $x \in L$ (a \textsc{YES}-instance),
  \item there exists a certificate (or witness) that can be verified in
  polynomial time by a deterministic Turing machine.
\end{itemize}

Thus, $\mathbf{NP}$ focuses on problems where \textsc{YES} answers are efficiently
verifiable.

\section*{Definition of the Class Co-NP}

\subsection*{Formal Definition}

A decision problem $L$ belongs to the class $\mathbf{Co\text{-}NP}$ if and only if
its complement belongs to $\mathbf{NP}$.

Formally,
\[
L \in \mathbf{Co\text{-}NP}
\quad \Longleftrightarrow \quad
\overline{L} \in \mathbf{NP},
\]
where
\[
\overline{L} = \{\, x \mid x \notin L \,\}.
\]

\subsection*{Equivalent Interpretation}

A problem is in $\mathbf{Co\text{-}NP}$ if:
\begin{itemize}
  \item for every \textsc{NO}-instance, there exists a certificate,
  \item that can be verified in polynomial time.
\end{itemize}

In other words, $\mathbf{Co\text{-}NP}$ is the class of problems whose \textsc{NO}
answers are efficiently verifiable.

\section*{Intuition Behind Co-NP}

\begin{itemize}
  \item $\mathbf{NP}$ problems: \emph{``If the answer is YES, I can quickly verify why.''}
  \item $\mathbf{Co\text{-}NP}$ problems: \emph{``If the answer is NO, I can quickly verify why.''}
\end{itemize}

Co-NP captures problems where it is difficult to prove that something exists, but
easy to prove that it does not exist.

\section*{Relationship Between NP and Co-NP}

Every problem in $\mathbf{P}$ belongs to both $\mathbf{NP}$ and $\mathbf{Co\text{-}NP}$:
\[
\mathbf{P} \subseteq \mathbf{NP} \cap \mathbf{Co\text{-}NP}.
\]

It is unknown whether:
\[
\mathbf{NP} = \mathbf{Co\text{-}NP}.
\]

Most complexity theorists believe:
\[
\mathbf{NP} \neq \mathbf{Co\text{-}NP},
\]
although this has not been proven. This question is closely related to the famous
$\mathbf{P}$ vs.\ $\mathbf{NP}$ problem.

\section*{Comparison of Complexity Classes: P, NP, and Co-NP}

\begin{center}
\begin{tabular}{|c|c|c|c|}
\hline
\textbf{Class} & \textbf{Certificate Verified} & \textbf{Verification Time} & \textbf{Example Problem} \\
\hline
P & YES or NO & Polynomial & Shortest Path \\
\hline
NP & YES instance & Polynomial & SAT \\
\hline
Co-NP & NO instance & Polynomial & UNSAT \\
\hline
\end{tabular}
\end{center}

\subsection*{Interpretation}

\begin{itemize}
  \item Problems in \textbf{P} can be solved directly in polynomial time.
  \item Problems in \textbf{NP} may be hard to solve, but YES answers are easy to verify.
  \item Problems in \textbf{Co-NP} may be hard to solve, but NO answers are easy to verify.
\end{itemize}
\section*{Types of Problems in Co-NP}

Problems in $\mathbf{Co\text{-}NP}$ typically involve \emph{universal claims}, such as:
\begin{itemize}
  \item ``No solution exists,''
  \item ``Every possible configuration satisfies a property,''
  \item ``A structure is valid for all cases.''
\end{itemize}

These problems are often complements of well-known $\mathbf{NP}$ problems.

\section*{Canonical Examples of Co-NP Problems}

\subsection*{TAUTOLOGY}

\textbf{Problem:} Given a Boolean formula $\varphi$, is $\varphi$ true under all
truth assignments?

The complement of this problem is \textsc{SAT}. Since \textsc{SAT} is in
$\mathbf{NP}$, \textsc{TAUTOLOGY} is in $\mathbf{Co\text{-}NP}$ and is
$\mathbf{Co\text{-}NP}$-complete.

\subsection*{UNSAT (Unsatisfiability)}

\textbf{Problem:} Given a Boolean formula $\varphi$, is it unsatisfiable?

Since \textsc{UNSAT} is the complement of \textsc{SAT}, and \textsc{SAT} is
$\mathbf{NP}$-complete, \textsc{UNSAT} is $\mathbf{Co\text{-}NP}$-complete.

\subsection*{Graph Non-Hamiltonicity}

\textbf{Problem:} Given a graph $G$, does $G$ not contain a Hamiltonian cycle?

The Hamiltonian Cycle problem is $\mathbf{NP}$-complete, hence its complement lies
in $\mathbf{Co\text{-}NP}$.

\subsection*{Composite Number Verification}

\textbf{Problem:} Given a number $n$, is $n$ composite?

This problem is in $\mathbf{NP}$ (a non-trivial factor is a certificate). Its
complement, primality testing, lies in $\mathbf{Co\text{-}NP}$ and is also known to
be in $\mathbf{P}$.

\section*{Co-NP-Complete Problems}

A problem $L$ is $\mathbf{Co\text{-}NP}$-complete if:
\begin{itemize}
  \item $L \in \mathbf{Co\text{-}NP}$, and
  \item every problem in $\mathbf{Co\text{-}NP}$ can be reduced to $L$ in polynomial
  time.
\end{itemize}

Examples include:
\begin{itemize}
  \item \textsc{TAUTOLOGY},
  \item \textsc{UNSAT}.
\end{itemize}

\section*{How Co-NP-Completeness Is Proved}

To prove that a problem $L$ is \textbf{Co-NP-complete}, two conditions must be
satisfied:

\subsection*{1. Membership in Co-NP}

One must show that the complement problem $\overline{L}$ belongs to NP.
Equivalently, there must exist a polynomial-time verifiable certificate for
NO-instances of $L$.

\subsection*{2. Co-NP-Hardness}

One must show that every problem in Co-NP can be reduced to $L$ in polynomial
time.

In practice, this is done by:
\begin{itemize}
  \item taking a known Co-NP-complete problem (such as UNSAT),
  \item giving a polynomial-time reduction from it to $L$.
\end{itemize}

\subsection*{Key Observation}

Since Co-NP problems are complements of NP problems, Co-NP-completeness proofs
often rely on known NP-completeness results by complementing both the problem
and the reduction.

For example:
\begin{itemize}
  \item SAT is NP-complete,
  \item UNSAT is therefore Co-NP-complete.
\end{itemize}
\section*{Structural Properties of Co-NP}

\begin{itemize}
  \item $\mathbf{Co\text{-}NP}$ is closed under complement by definition.
  \item It is not known to be closed under union or intersection.
  \item If $\mathbf{NP} = \mathbf{Co\text{-}NP}$, the polynomial hierarchy collapses.
\end{itemize}

\section*{Final Conclusion}

$\mathbf{Co\text{-}NP}$ is the class of decision problems whose \textsc{NO}-instances
have polynomial-time verifiable certificates.

In summary:
\begin{itemize}
  \item $\mathbf{Co\text{-}NP}$ consists of complements of $\mathbf{NP}$ problems,
  \item it captures problems with efficiently verifiable non-existence proofs,
  \item it contains problems such as \textsc{UNSAT} and \textsc{TAUTOLOGY},
  \item whether $\mathbf{Co\text{-}NP} = \mathbf{NP}$ remains an open question.
\end{itemize}


\paragraph{Polynomial Hierarchy Remark.}
If NP = Co-NP, then the polynomial hierarchy collapses to its first level.


\section*{Intuition}

$\mathbf{NP}$ verifies existence, whereas $\mathbf{Co\text{-}NP}$ verifies
non-existence.
\end{document}
