\documentclass{article}

% --------------------
% Packages
% --------------------
\usepackage[utf8]{inputenc}
\usepackage{amsmath, amssymb}
\usepackage{graphicx}
\usepackage{geometry}

% --------------------
% Page Layout
% --------------------
\geometry{
  top=1in,
  bottom=1in,
  left=1in,
  right=1in
}

\setlength{\parindent}{0pt}
\setlength{\parskip}{1em}

% --------------------
% Document
% --------------------
\begin{document}
\vspace*{-1.5cm}

% --------------------
% Header Section
% --------------------
\hspace{0.7cm}
\begin{minipage}[c]{0.7\textwidth}
\vspace{0.5cm}
\textbf{MTech CSE – 1st Semester} \\  
Student ID: \textbf{A125023} \\  
Student Name: \textbf{SRIJITA VERMA}
\end{minipage}
\hspace{0.5cm}
\begin{minipage}{0.2\textwidth}
\centering
\includegraphics[width=0.6\linewidth]{college_logo.jpg}
\end{minipage}

\vspace{1cm}

% --------------------
% Question Section
% --------------------
\section*{Question}

Is the \textsc{3-SAT} (3-CNF-SAT) problem NP-hard? Justify your answer.

% --------------------
% Answer Section
% --------------------
\section*{Answer}
\section*{Short Answer}

Yes, the \textsc{3-SAT} (3-CNF-SAT) problem is NP-hard.  
In fact, \textsc{3-SAT} is NP-complete.

\section*{Background and Definitions}

\subsection*{Boolean Satisfiability (SAT)}

The \textsc{SAT} problem asks whether a given Boolean formula has a truth
assignment that makes the formula evaluate to \textsc{true}. \textsc{SAT} was the
first problem proven to be NP-complete, as shown by the Cook--Levin Theorem.

\subsection*{3-SAT (3-CNF-SAT)}

In \textsc{3-SAT}, the Boolean formula is restricted to:
\begin{itemize}
  \item Conjunctive Normal Form (CNF), and
  \item each clause contains exactly three literals.
\end{itemize}

Formally, a \textsc{3-SAT} formula has the form:
\[
\varphi = \bigwedge_{i=1}^{m}
\left( \ell_{i1} \,\vee\, \ell_{i2} \,\vee\, \ell_{i3} \right),
\]
where each literal $\ell_{ij}$ is either a variable or its negation.

\section*{Meaning of NP-Hardness}

A problem $P$ is \emph{NP-hard} if every problem in $\mathbf{NP}$ can be reduced
to $P$ in polynomial time. If a problem is both NP-hard and belongs to
$\mathbf{NP}$, then it is \emph{NP-complete}.

\section*{Membership of 3-SAT in NP}

Given a truth assignment:
\begin{itemize}
  \item each clause can be checked in constant time,
  \item all clauses can be checked in linear time.
\end{itemize}

Thus, a satisfying assignment is a polynomial-time verifiable certificate.
Therefore,
\[
\textsc{3-SAT} \in \mathbf{NP}.
\]

\section*{Polynomial Reduction from SAT to 3-SAT}

To establish NP-hardness, we show that:
\[
\textsc{SAT} \leq_p \textsc{3-SAT}.
\]

That is, any CNF formula can be transformed into an equivalent 3-CNF formula in
polynomial time.

\subsection*{Clause Transformation}

Consider a CNF clause of arbitrary length:
\[
(x_1 \vee x_2 \vee x_3 \vee \cdots \vee x_k).
\]

This clause can be replaced by a conjunction of 3-literal clauses by introducing
new variables:
\[
(x_1 \vee x_2 \vee y_1)
\;\wedge\;
(\neg y_1 \vee x_3 \vee y_2)
\;\wedge\;
\cdots
\;\wedge\;
(\neg y_{k-3} \vee x_{k-1} \vee x_k).
\]

This transformation satisfies the following properties:
\begin{itemize}
  \item satisfiability is preserved,
  \item the number of new variables and clauses grows linearly,
  \item the transformation runs in polynomial time.
\end{itemize}

\subsection*{Worked Reduction Example: SAT to 3-SAT}

Consider the CNF formula:
\[
\varphi =
(x_1 \vee x_2 \vee x_3 \vee x_4)
\;\wedge\;
(\neg x_1 \vee x_2)
\;\wedge\;
(x_3).
\]

This formula is in CNF, but not all clauses contain exactly three literals. We now
convert it into an equivalent 3-CNF formula.

\subsection*{Step 1: Handling Clauses with More Than Three Literals}

The clause
\[
(x_1 \vee x_2 \vee x_3 \vee x_4)
\]
contains four literals. Introduce a new variable $y_1$ and rewrite it as:
\[
(x_1 \vee x_2 \vee y_1)
\;\wedge\;
(\neg y_1 \vee x_3 \vee x_4).
\]

This transformation preserves satisfiability:
\begin{itemize}
  \item if the original clause is satisfiable, the new clauses are satisfiable;
  \item if the new clauses are satisfiable, at least one of the original literals
  must be true.
\end{itemize}

\subsection*{Step 2: Handling Clauses with Fewer Than Three Literals}

The clause
\[
(\neg x_1 \vee x_2)
\]
contains two literals. We duplicate one literal to obtain:
\[
(\neg x_1 \vee x_2 \vee x_2).
\]

This duplication does not change the logical meaning of the clause.

\subsection*{Step 3: Handling Single-Literal Clauses}

The clause
\[
(x_3)
\]
contains a single literal. We duplicate it to obtain:
\[
(x_3 \vee x_3 \vee x_3).
\]

Again, satisfiability is preserved.

\subsection*{Step 4: Final 3-CNF Formula}

The resulting 3-CNF formula is:
\[
\varphi' =
(x_1 \vee x_2 \vee y_1)
\;\wedge\;
(\neg y_1 \vee x_3 \vee x_4)
\;\wedge\;
(\neg x_1 \vee x_2 \vee x_2)
\;\wedge\;
(x_3 \vee x_3 \vee x_3).
\]

The formula $\varphi'$ is satisfiable if and only if the original formula
$\varphi$ is satisfiable.

\subsection*{Step 5: Complexity of the Reduction}

Each clause is transformed using a constant number of new variables and clauses.
The total size of the formula increases linearly, and the transformation can be
performed in polynomial time.

Therefore,
\[
\textsc{SAT} \leq_p \textsc{3-SAT}.
\]
\section*{Consequence}

Since:
\begin{itemize}
  \item \textsc{SAT} is NP-complete, and
  \item \textsc{SAT} reduces to \textsc{3-SAT} in polynomial time,
\end{itemize}
we conclude that:
\[
\textsc{3-SAT} \text{ is NP-hard}.
\]

\section*{Final Classification}

We have shown that:
\[
\textsc{3-SAT} \in \mathbf{NP}
\quad \text{and} \quad
\textsc{SAT} \leq_p \textsc{3-SAT}.
\]

Therefore,
\[
\textsc{3-SAT} \text{ is NP-complete},
\]
and hence NP-hard.

\section*{Why This Result Is Important}

The NP-hardness of \textsc{3-SAT} is fundamental because:
\begin{itemize}
  \item many NP-hardness proofs reduce from \textsc{3-SAT},
  \item it serves as a standard starting point for reductions,
  \item it shows that even very restricted Boolean formulas remain computationally hard.
\end{itemize}

Despite each clause having only three literals, the problem retains the full
difficulty of \textsc{SAT}.

\section*{Intuition}

\textsc{SAT} is hard because it encodes arbitrary logical constraints. Restricting
clauses to length three does not reduce expressive power; it only standardizes the
structure. Thus, \textsc{3-SAT} captures the essence of NP-hardness in a highly
controlled form.

\section*{Final Remark}

Although \textsc{3-SAT} is NP-hard, special cases such as \textsc{2-SAT} are
solvable in polynomial time. This highlights how small syntactic restrictions can
dramatically change computational complexity.

\end{document}
