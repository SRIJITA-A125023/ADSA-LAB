\documentclass{article}

% --------------------
% Packages
% --------------------
\usepackage[utf8]{inputenc}
\usepackage{amsmath, amssymb}
\usepackage{graphicx}
\usepackage{geometry}

% --------------------
% Page Layout
% --------------------
\geometry{
  top=1in,
  bottom=1in,
  left=1in,
  right=1in
}

\setlength{\parindent}{0pt}
\setlength{\parskip}{1em}

% --------------------
% Document
% --------------------
\begin{document}
\vspace*{-1.5cm}

% --------------------
% Header Section
% --------------------
\hspace{0.7cm}
\begin{minipage}[c]{0.7\textwidth}
\vspace{0.5cm}
\textbf{MTech CSE – 1st Semester} \\  
Student ID: \textbf{A125023} \\  
Student Name: \textbf{SRIJITA VERMA}
\end{minipage}
\hspace{0.5cm}
\begin{minipage}{0.2\textwidth}
\centering
\includegraphics[width=0.6\linewidth]{college_logo.jpg}
\end{minipage}

\vspace{1cm}

% --------------------
% Question Section
% --------------------
\section*{Question}
Prove that every connected component of the symmetric difference of two
matchings in a graph $G$ is either a path or an even-length cycle.

% --------------------
% Answer Section
% --------------------
\section*{Answer}
\section*{Preliminaries and Definitions}

Let $G = (V,E)$ be an undirected graph.

\subsection*{Definition 1: Matching}

A matching $M \subseteq E$ is a set of edges such that no two edges in $M$ share a
common endpoint.

Let $M_1$ and $M_2$ be two matchings in $G$.

\subsection*{Definition 2: Symmetric Difference}

The symmetric difference of $M_1$ and $M_2$ is defined as:
\[
M_1 \oplus M_2 = (M_1 \setminus M_2) \cup (M_2 \setminus M_1).
\]

That is, $M_1 \oplus M_2$ consists of edges that belong to exactly one of the two
matchings.

\section*{Goal}

We must show that every connected component of the graph
\[
H = (V,\, M_1 \oplus M_2)
\]
is either:
\begin{itemize}
  \item a path, or
  \item a cycle of even length.
\end{itemize}

\section*{Key Observations}

\subsection*{Observation 1: Degree Bound}

Fix any vertex $v \in V$.

Since $M_1$ is a matching, at most one edge of $M_1$ is incident to $v$.
Since $M_2$ is a matching, at most one edge of $M_2$ is incident to $v$.

Therefore, in the symmetric difference $M_1 \oplus M_2$,
\[
\deg_H(v) \le 2.
\]

\subsection*{Observation 2: Structure of Graphs with Maximum Degree 2}

Any graph in which every vertex has degree at most $2$ consists of connected
components that are:
\begin{itemize}
  \item paths, or
  \item cycles.
\end{itemize}

Thus, each connected component of $H$ must be either a path or a cycle. It
remains to prove that every cycle has even length.

\section*{Alternating Structure of the Symmetric Difference}

\subsection*{Observation 3: Alternation of Edges}

Consider any vertex $v$ with $\deg_H(v) = 2$.

One incident edge must belong to $M_1$, and the other must belong to $M_2$.
This is because no matching can contribute more than one incident edge at $v$.

Hence, along any connected component of $H$, edges alternate between:
\[
M_1,\, M_2,\, M_1,\, M_2,\, \ldots
\]

\section*{Analysis of Connected Components}

\subsection*{Case 1: The Component Is a Path}

If a connected component contains a vertex of degree $1$, then it is a path.
Such a path may start and end at vertices that are unmatched in one or both
matchings.

No restriction is imposed on the length of such a path. Hence, paths are valid
connected components.

\subsection*{Case 2: The Component Is a Cycle}

Suppose a connected component is a cycle. Then every vertex on the cycle has
degree exactly $2$.

Edges on the cycle alternate between $M_1$ and $M_2$. Let the cycle have length
$k$.

Since the edges alternate, exactly half of the edges belong to $M_1$ and half to
$M_2$. This is possible only if $k$ is even.

Therefore, every cycle in $M_1 \oplus M_2$ has even length.

\section*{Final Conclusion}

We have shown that:
\begin{itemize}
  \item every vertex in $M_1 \oplus M_2$ has degree at most $2$,
  \item hence, every connected component is either a path or a cycle,
  \item all cycles must be of even length due to edge alternation.
\end{itemize}

Thus, every connected component of $M_1 \oplus M_2$ is either a path or an
even-length cycle.

\section*{Worked Example (Textual Diagram)}

Consider the graph with vertices:
\[
V = \{a, b, c, d, e\}.
\]

Let the two matchings be:
\[
M_1 = \{(a,b), (c,d)\}, \quad
M_2 = \{(b,c), (d,e)\}.
\]

\subsection*{Textual Diagram Representation}

We describe the graph structure textually:

\begin{center}
$a$ --- $b$ --- $c$ --- $d$ --- $e$
\end{center}

Edges in the matchings:
\begin{itemize}
  \item Edges in $M_1$: $(a,b)$ and $(c,d)$,
  \item Edges in $M_2$: $(b,c)$ and $(d,e)$.
\end{itemize}

The symmetric difference is:
\[
M_1 \oplus M_2 = \{(a,b), (b,c), (c,d), (d,e)\}.
\]

\subsection*{Resulting Component}

The symmetric difference forms a single connected component:
\[
a \rightarrow b \rightarrow c \rightarrow d \rightarrow e,
\]
which is a \emph{path}.

The edges alternate between $M_1$ and $M_2$:
\[
M_1,\, M_2,\, M_1,\, M_2.
\]

This example illustrates how a connected component of the symmetric difference
can form a path with alternating edges.
\section*{Worked Example (Even-Length Cycle)}

Consider the graph with vertices:
\[
V = \{v_1, v_2, v_3, v_4\}.
\]

Let the two matchings be:
\[
M_1 = \{(v_1,v_2), (v_3,v_4)\}, \quad
M_2 = \{(v_2,v_3), (v_4,v_1)\}.
\]

\subsection*{Textual Diagram Representation}

The graph can be described as a cycle:
\begin{center}
$v_1$ --- $v_2$ --- $v_3$ --- $v_4$ --- $v_1$
\end{center}

Edges alternate between the two matchings around the cycle.

\subsection*{Symmetric Difference}

Since no edge is common to both matchings, the symmetric difference is:
\[
M_1 \oplus M_2 =
\{(v_1,v_2), (v_2,v_3), (v_3,v_4), (v_4,v_1)\}.
\]

This forms a single connected component that is a cycle of length $4$.

\subsection*{Observation}

The edges alternate between $M_1$ and $M_2$, and the cycle length is even. This
illustrates the second possible structure of a connected component in the
symmetric difference.
\section*{Relation to Augmenting Paths and Maximum Matching}

This structural result has a direct and fundamental connection to augmenting
paths in matching theory.

Let $M$ be a matching and $M^\ast$ be a maximum matching in a graph $G$. Consider
the symmetric difference:
\[
M \oplus M^\ast.
\]

By the result proved above, every connected component of $M \oplus M^\ast$ is
either a path or an even-length cycle with alternating edges.

\subsection*{Augmenting Paths}

An \emph{augmenting path} is a path that:
\begin{itemize}
  \item starts and ends at vertices unmatched by $M$,
  \item alternates between edges not in $M$ and edges in $M$,
  \item begins and ends with edges not in $M$.
\end{itemize}

In $M \oplus M^\ast$, such augmenting paths correspond exactly to path components
whose endpoints are unmatched in $M$.

Flipping the matching along such a path increases the size of the matching by
one.

\subsection*{Even-Length Cycles}

Even-length cycle components in $M \oplus M^\ast$:
\begin{itemize}
  \item alternate between $M$ and $M^\ast$,
  \item do not change the size of the matching when flipped,
  \item represent different but equally sized matchings.
\end{itemize}

Thus:
\begin{itemize}
  \item path components explain how a matching can be improved,
  \item cycle components explain structural differences between maximum
        matchings.
\end{itemize}

This decomposition underlies the correctness of classical matching algorithms,
including augmenting-path-based algorithms for maximum matching.


\section*{Intuition}

Matchings pair vertices without overlap. Taking the symmetric difference
highlights exactly where the two matchings disagree. At these disagreement
points, edges alternate cleanly between the two matchings, forcing the structure
to be either paths or even cycles—and nothing else.
\subsection*{Remark (Bipartite Graphs)}
In bipartite graphs, all alternating cycles are even, which aligns naturally with
the structure of the symmetric difference.

\end{document}
