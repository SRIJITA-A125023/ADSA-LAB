\documentclass{article}

% --------------------
% Packages
% --------------------
\usepackage[utf8]{inputenc}
\usepackage{amsmath, amssymb}
\usepackage{graphicx}
\usepackage{geometry}

% --------------------
% Page Layout
% --------------------
\geometry{
  top=1in,
  bottom=1in,
  left=1in,
  right=1in
}

\setlength{\parindent}{0pt}
\setlength{\parskip}{1em}

% --------------------
% Document
% --------------------
\begin{document}
\vspace*{-1.5cm}

% --------------------
% Header Section
% --------------------
\hspace{0.7cm}
\begin{minipage}[c]{0.7\textwidth}
\vspace{0.5cm}
\textbf{MTech CSE – 1st Semester} \\  
Student ID: \textbf{A125023} \\  
Student Name: \textbf{SRIJITA VERMA}
\end{minipage}
\hspace{0.5cm}
\begin{minipage}{0.2\textwidth}
\centering
\includegraphics[width=0.6\linewidth]{college_logo.jpg}
\end{minipage}

\vspace{1cm}

% --------------------
% Question Section
% --------------------
\section*{Question}

For finding an augmenting path in a graph, should Breadth-First Search (BFS) or
Depth-First Search (DFS) be applied? Justify your answer.
% --------------------
% Answer Section
% --------------------
\section*{Answer}
\section*{Short Answer}

Breadth First Search (BFS) should be applied when finding augmenting paths in the
context of network flow algorithms, particularly in the Edmonds--Karp algorithm.

Although Depth First Search (DFS) can be used to find augmenting paths, BFS is
preferred because it guarantees shortest augmenting paths, ensures
polynomial-time complexity, and avoids pathological performance cases.

\section*{Background: Augmenting Paths in Flow Networks}

An augmenting path is a path from the source to the sink in the residual graph
along which additional flow can be pushed.

Augmenting paths are central to:
\begin{itemize}
  \item the Ford--Fulkerson method,
  \item the Edmonds--Karp algorithm,
  \item maximum flow computation.
\end{itemize}

The choice of how augmenting paths are found (BFS vs.\ DFS) directly affects
correctness guarantees and time complexity.

\section*{BFS vs DFS: Conceptual Comparison}

\subsection*{Depth First Search (DFS)}

\begin{itemize}
  \item Explores one path as deeply as possible before backtracking.
  \item May find very long or inefficient augmenting paths.
  \item The choice of path depends heavily on traversal order.
  \item Does not guarantee the shortest augmenting path.
\end{itemize}

\subsection*{Breadth First Search (BFS)}

\begin{itemize}
  \item Explores paths in increasing order of length (number of edges).
  \item Always finds the shortest augmenting path.
  \item Uses systematic, level-by-level exploration.
  \item Exhibits predictable and stable behavior.
\end{itemize}

\section*{Why BFS Is Preferred for Finding Augmenting Paths}

\subsection*{1. Shortest Augmenting Path Property}

When BFS is used on the residual graph, the first time the sink is reached, the
path found has the minimum number of edges. Shorter augmenting paths reduce the
number of augmentations required.

This property is essential in the Edmonds--Karp algorithm.

\subsection*{2. Guaranteed Polynomial-Time Complexity}

Using DFS in the Ford--Fulkerson method can lead to exponential time complexity
in the worst case. In contrast, using BFS guarantees a polynomial-time bound.

Specifically, the Edmonds--Karp algorithm runs in:
\[
O(VE^2)
\]
time, where $V$ is the number of vertices and $E$ is the number of edges.

\subsection*{3. Avoidance of Bad Augmenting Paths}

DFS may:
\begin{itemize}
  \item repeatedly push small amounts of flow,
  \item undo previous flow decisions,
  \item become trapped in inefficient augmentations.
\end{itemize}

BFS avoids these issues by:
\begin{itemize}
  \item making steady progress using shortest paths,
  \item ensuring that the distance from source to sink never decreases.
\end{itemize}

\subsection*{4. Theoretical Guarantee of Termination}

With BFS:
\begin{itemize}
  \item each edge can become critical only a bounded number of times,
  \item the level of the sink strictly increases after enough augmentations.
\end{itemize}

This provides a provable upper bound on the number of iterations.

\section*{Illustrative Example: BFS vs DFS for Augmenting Paths}

Consider the following flow network:

\[
s \rightarrow v_1 \rightarrow v_2 \rightarrow t
\]
\[
s \rightarrow v_3 \rightarrow t
\]

Assume all edges have unit capacity.

\subsection*{Using DFS}

A depth-first search may discover the augmenting path:
\[
s \rightarrow v_1 \rightarrow v_2 \rightarrow t
\]
which has length 3.

After pushing one unit of flow, DFS may repeatedly explore long paths or undo
previous choices before discovering alternative shorter routes.

\subsection*{Using BFS}

Breadth-first search explores the graph level by level and finds the augmenting
path:
\[
s \rightarrow v_3 \rightarrow t
\]
which has length 2.

This is the shortest augmenting path in terms of number of edges and leads to
faster convergence.

\subsection*{Observation}

Although both searches find valid augmenting paths, BFS consistently selects
shorter paths, whereas DFS may choose longer and inefficient paths depending on
the traversal order.
\section*{Formal Justification (Edmonds--Karp Insight)}

In the Edmonds--Karp algorithm:
\begin{itemize}
  \item BFS is used to find augmenting paths,
  \item each BFS takes $O( O(E) )$ time,
  \item the number of augmentations is $O(VE)$.
\end{itemize}

Hence, the total running time is:
\[
O(VE^2).
\]

This guarantee does not hold if DFS is used instead.

\section*{Can DFS Ever Be Used?}

Yes, DFS can be used:
\begin{itemize}
  \item in the basic Ford--Fulkerson algorithm,
  \item for small graphs,
  \item when capacities are integers and inputs are well-behaved.
\end{itemize}

However:
\begin{itemize}
  \item it offers no worst-case time guarantees,
  \item it is unsuitable for large or adversarial inputs.
\end{itemize}

\section*{Relation to Ford--Fulkerson and Edmonds--Karp Algorithms}

The choice between BFS and DFS directly distinguishes the
Ford--Fulkerson and Edmonds--Karp algorithms.

\subsection*{Ford--Fulkerson Method}

The Ford--Fulkerson method allows augmenting paths to be found using
\emph{any} graph traversal strategy, including DFS.

As a result:
\begin{itemize}
  \item the algorithm is correct,
  \item but the running time depends on the choice of augmenting paths,
  \item and in the worst case, it may take exponential time.
\end{itemize}

\subsection*{Edmonds--Karp Algorithm}

The Edmonds--Karp algorithm is a specific implementation of
Ford--Fulkerson that uses BFS to find augmenting paths.

By always selecting the shortest augmenting path:
\begin{itemize}
  \item the number of augmentations is bounded,
  \item the distance from source to sink never decreases,
  \item the algorithm runs in polynomial time.
\end{itemize}

Specifically, Edmonds--Karp runs in:
\[
O(VE^2)
\]
time.

\subsection*{Key Distinction}

Thus, while DFS can be used in Ford--Fulkerson, BFS is essential in
Edmonds--Karp to guarantee efficiency and predictable performance.

\section*{Final Conclusion}

Breadth First Search (BFS) should be used to find augmenting paths.

\textbf{Reasons:}
\begin{itemize}
  \item BFS finds the shortest augmenting paths,
  \item guarantees polynomial-time performance,
  \item prevents inefficient or cyclic augmentations,
  \item forms the basis of the Edmonds--Karp algorithm.
\end{itemize}
\paragraph{Practical Consideration.}
In practice, BFS-based augmenting path selection leads to more stable flow growth
and fewer iterations, making it suitable for large networks.

\section*{Intuition}

Augmenting flow is like sending water through pipes. BFS finds the shortest route
to the sink. Shorter routes fill faster and stabilize sooner, whereas DFS may
wander unnecessarily and slow convergence.

\end{document}
